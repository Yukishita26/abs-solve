\def\print#1{\immediate\write16{#1}}%
\def\substitute#1 #2 #3;{\def\inN{#1} \def\inA{#2} \def\inB{#3}}%
%\def\blank{ }
%
\read-1to\inLine%
\expandafter\substitute\inLine;%
\newcount\cntA \cntA=\inA \advance\cntA by -1%
\newcount\cntB \cntB=\inB \advance\cntB by 1% A<=k<=B ⇔ A-1<k<B+1
\newcount\cntN \cntN=\inN \advance\cntN by 1%
\newcount\cnti \cnti=1%
\newcount\tmp \newcount\result \result=0%
\newcount\disitA \newcount\disitB \newcount\disitC \newcount\disitD \newcount\disitE%
\loop \ifnum \cnti<\cntN%
  %高々5桁なので5桁調べる
  \disitA=\cnti \divide\disitA by 10000 \multiply\disitA by 10000%
  \disitB=\cnti \advance\disitB by -\disitA \divide\disitB by 1000 \multiply\disitB by 1000%
  \disitC=\cnti \advance\disitC by -\disitA \advance\disitC by -\disitB \divide\disitC by 100 \multiply\disitC by 100%
  \disitD=\cnti \advance\disitD by -\disitA \advance\disitD by -\disitB \advance\disitD by -\disitC \divide\disitD by 10 \multiply\disitD by 10%
  \disitE=\cnti \advance\disitE by -\disitA \advance\disitE by -\disitC \advance\disitE by -\disitC \advance\disitE by -\disitD%
  \divide\disitA by 10000 \divide\disitB by 1000 \divide\disitC by 100 \divide\disitD by 10%
  %\print{\the\disitA\blank \the\disitB\blank \the\disitC\blank \the\disitD\blank \the\disitE\blank}
  \tmp=\disitA \advance\tmp by \disitB \advance\tmp by \disitC \advance\tmp by \disitD \advance\tmp by \disitE%
  \ifnum \cntA<\tmp%
    \ifnum \tmp<\cntB%
      \advance\result by \cnti%
    \fi%
  \fi%
\advance\cnti by 1 \repeat%
\print{\the\result}
\end%