\documentclass{article}
\usepackage{expl3}
\begin{document}
\ExplSyntaxOn

% 入力をtlで読み込み
\tl_new:N \l__A_tl
\tl_new:N \l__B_tl
\tl_new:N \l__C_tl
\tl_new:N \l__X_tl
\ior_str_get_term:nN {} \l__A_tl
\ior_str_get_term:nN {} \l__B_tl
\ior_str_get_term:nN {} \l__C_tl
\ior_str_get_term:nN {} \l__X_tl

\int_new:N \l__result_int
\int_set:Nn \l__result_int { 0 }
% 0<=a<=A, 0<=b<=B, 0<=c<=C の3重ループ
\int_new:N \l__a_counter_int
\int_new:N \l__b_counter_int
\int_new:N \l__c_counter_int
\int_set:Nn \l__a_counter_int { 0 }
\int_do_while:nn { \l__a_counter_int <= \l__A_tl } {
	\int_set:Nn \l__b_counter_int { 0 }
	\int_do_while:nn { \l__b_counter_int <= \l__B_tl } {
		\int_set:Nn \l__c_counter_int { 0 }
		\int_do_while:nn { \l__c_counter_int <= \l__C_tl } {
			\int_add:Nn \l__result_int {
				% 500a+100b+50c=X なら1を,そうでなければ0をresultに足す 
				\int_compare:nNnTF {
					 \l__a_counter_int * 500 + \l__b_counter_int * 100 + \l__c_counter_int * 50
				} = { \l__X_tl } { 1 } { 0 }
			}
			\int_incr:N \l__c_counter_int
		}
		\int_incr:N \l__b_counter_int
	}
	\int_incr:N \l__a_counter_int
}
\iow_term:x { \int_eval:n{ \l__result_int } }



%\cs_new:Npn \_check_abc:nnn #1 #2 #3 {
%	\int_compare:nNnTF { #1 * 500 + #2 * 100 + #3 * 50} = { \__X_tl } {
%		1
%	}{
%		0
%	}
%}
%\cs_new:Npn \_c_loop:nn #1 #2 {
%	\int_new:N \l_restult_c
%	\int_set:Nn \l_result_c { 0 }
%	\cs_new:Npn \_c_loop_sub:n #3 {
%		\int_add:Nn \l_result_c { \_check_abc:nnn { #1 } { #2 } { #3 } }
%	}
%	\int_step_function:nnN { 0 } { \l__C_tl } \_c_loop_sub:n
%}
%
%\iow_term:x { \_c_loop:nn { 0 } { 0 } }


\ExplSyntaxOff
\end{document}